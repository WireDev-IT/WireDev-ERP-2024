%%%%%%%%%%%%%%%%%%%%%%%%%%%%%%%%%%%%%%%%%%%%%%%%%%%%%%%%%%%%%
%   LaTeX-Arbeitsvorlage, Version 5.15                	  	%
%   Autor: 		Mirco Lukas <http://mircol.de>              %
%   Lizenz: 	The MIT License (MIT)						%
%   Updates: 	https://github.com/MircoL/LatexTemplateDE   %
%%%%%%%%%%%%%%%%%%%%%%%%%%%%%%%%%%%%%%%%%%%%%%%%%%%%%%%%%%%%%

\RequirePackage{xstring}

\makeatletter
	\@ifundefined{dokumentTyp}{%
		\errmessage{DokumentTyp muss angegeben werden!} %
		\stop %
	}
\makeatother

\IfStrEqCase{\dokumentTyp}{
	{Skript}{\documentclass[12pt,ngerman,openany,numbers=noenddot,listof=totoc,bibliography=totoc,oneside]{scrbook}}%
	{Thesis}{\documentclass[12pt,ngerman,openany,numbers=noenddot,listof=totoc,bibliography=totoc,oneside]{scrbook}}%
	{Mitschrift}{\documentclass[a4paper,12pt,ngerman,oneside]{scrartcl}}%
	{Praesentation}{\documentclass{beamer}}%
}[\errmessage{Ungueltiger Dokumenttyp: \dokumentTyp}\stop]


% =====================================================
%    Kompatibilitätseinstellungen für ältere Versionen
% =====================================================
% V5.9+	\providecommand{\hauptsprache}{ngerman}
	\providecommand{\andereSprachen}{english,polish}

% =====================================================
%    Pakete
% =====================================================

% sollte möglichst weit oben stehen
\usepackage{morewrites}

\usepackage{array}

\usepackage[xcolor,pdftex,rightbars]{changebar}

\usepackage{graphicx}
	\usepackage{epstopdf}

\usepackage{amsmath}

\usepackage{amssymb}

\usepackage[main=\hauptsprache,\andereSprachen]{babel}

% Tabelle farbig markieren: \columncolor{wunschfarbe} \rowcolor{}, \cellcolor{}
\usepackage{colortbl}

\usepackage{enumitem}

\usepackage{environ}

\usepackage{esint}

\usepackage{eurosym}

\usepackage[OT2, T2A,TS1,T1]{fontenc}


\usepackage[utf8]{inputenc}

\usepackage{listings}

\usepackage{lipsum}

\usepackage{ifthen}

\usepackage{ifsym}

% erlaubt mehrseitige Tabellen:  \begin{longtable}{|l|r|c|p{2cm}|} \hline ... \end{longtable}
\usepackage{longtable}

\usepackage{mVersion}
\increaseBuild

\usepackage{makeidx}

\usepackage{multicol}

% Workaround for Linux with texlive

\providecommand{\oeschfamily}{\em}
\usepackage{pgfkeys}

\usepackage{polski}

% m u s s  vor hyperref stehen!o
\providecommand{\zeilenabstand}{singlespacing}
\usepackage[\zeilenabstand]{setspace}

\usepackage{soul}

\usepackage{tcolorbox}



\usepackage[\hauptsprache]{varioref}

\usepackage{wasysym}

\usepackage{xspace}

\usepackage{xargs}

\usepackage{yfonts}
% Lade Pakete abhängig vom DokumentTyp.
\IfStrEqCase{\dokumentTyp}{
	{Skript}{
		\usepackage[a4paper]{geometry}
		\usepackage[hyphens]{url}
		\usepackage[pdftex,pdfborderstyle={/S/U/W 1}]{hyperref}
	}
	{Thesis}{
		\usepackage[a4paper]{geometry}
		\usepackage[hyphens]{url}
		\usepackage[pdftex,pdfborderstyle={/S/U/W 1}]{hyperref}
	}
	{Mitschrift}{
		\usepackage{fancyhdr}
		\usepackage[a4paper]{geometry}
		\usepackage[hyphens]{url}
		\usepackage[pdftex,pdfborderstyle={/S/U/W 1}]{hyperref}
		\let\chapter\section
	}
	{Praesentation}{

		\usecolortheme{whale}

		\newcommand{\fip}[1][]{\frametitle{\insertsection\xspace##1}\pause}
		\newcommand{\finp}[1][]{\frametitle{\insertsection\xspace##1}}

		\usefonttheme{professionalfonts}

		\setbeamercovered{invisible}  % Overlays auf den Folien

		\setbeamertemplate{footline}{
			\begin{beamercolorbox}[sep=1.8em,wd=\paperwidth,leftskip=0.5cm,rightskip=0.5cm]{footlinecolor}
			\end{beamercolorbox}%
			\insertsectionnavigationhorizontal{\paperwidth}{}{\hskip0pt plus1filll}
		}

		\setbeamertemplate{footline}{
			\begin{minipage}[c]{4cm}
				\get{Version 5/all/Titel}
			\end{minipage}
			\begin{minipage}[l]{4cm}
				\ifthenelse{\equal{\get{Version 5/Praesentationen/TitelKurz}}{}}{\get{Version 5/all/Titel}}{\get{Version 5/Praesentationen/TitelKurz}}
			\end{minipage}
			\begin{minipage}[c]{2cm}
				\StrSubstitute{\get{all/Autoren}}{&}{\par}
			\end{minipage}
			\hfill
			\begin{minipage}[c]{1.2cm}
				Folie \insertframenumber\ (\inserttotalframenumber)
			\end{minipage}
		}
	}
}

% m u s s   hier stehen!
\setlength{\marginparwidth}{2cm}
\usepackage{todonotes}


% =====================================================
%    Eigene Befehle
% =====================================================
\newcommand{\andd}{\wedge}
	\newcommand{\bigandd}{\bigwedge}

\newcommand{\aufgabe}[1]{%
	\command{\aktuelleAufgabe}{#1}%
	\section*{Aufgabe \aktuelleAufgabe}%
}
	\newcommand{\teilaufgabe}[1]{\paragraph*{\underline{TA \aktuelleAufgabe.#1}}}
	\newcommand{\aktuelleAufgabe}{1}

\newboolean{isAppendixSet}
	\let\appendixOld\appendix
	\renewcommand{\appendix}{\ifthenelse{\boolean{isAppendixSet}}{}{\appendixOld\setboolean{isAppendixSet}{true}}}

\newcommand{\blitz}{\lightning\xspace}

\newcommand{\countAutoren}[1]{\newcounter{autorenC}\setcounter{autorenC}{#1}}

% Nicht \dh, sonst Kollision mit dem Buchstaben ð (Eth)
\newcommand{\dhe}{d.\,h.\xspace}
\newcommand{\Dhe}{D.\,h.\xspace}

\newcommand{\ellipse}{.\hspace*{-.1em}.\hspace*{-.1em}.}

\newcommand{\fallunterscheidung}[3][]{\ensuremath{#1 \left\{\begin{array}{cl}#2\\#3\end{array}\right.}}
	\newcommand{\fuer}{&\text{für }}
	\newcommand{\sonst}{&\text{sonst}}

\newcommand{\get}[1]{\pgfkeysvalueof{/VorlageVersion5/#1}}

\newcommand{\idr}{i.\,d.\,R.\xspace}
\newcommand{\Idr}{I.\,d.\,R.\xspace}

\newcommand{\length}[1]{\ensuremath{|#1|}}

\newcommand{\name}[1]{\mbox{\textsc{#1}}}

\newcommand{\neuerbegriff}[1]{\tcbox[size=small,on line,colframe=red!75!black,colback=blue!5!white,fontupper=\normalsize]{#1}\xspace}

\newcommand{\neuerbegriffIdx}[2][]{\ifthenelse{\equal{#1}{}}{\index{#2}\neuerbegriff{#2}}{\index{#1}\neuerbegriff{#2}}}

\makeindex

\newcommand{\obda}{o.\,B.\,d.\,A.\xspace}
\newcommand{\Obda}{O.\,B.\,d.\,A.\xspace}

\makeatletter
	\def\colorref@i		{\text{\textmarker{yellow}{(i)}}}
	\def\colorref@ii	{\text{\textmarker{green}{(ii)}}}
	\def\colorref@iii	{\text{\textmarker{red}{(iii)}}}
	\def\colorref@iv	{\text{\textcolor{white}{\textmarker{blue}{(iv)}}}}
\makeatother
\newcounter{colorrefCtr}
\def\colorref(#1){ %
	\setcounter{colorrefCtr}{#1} %
	\makeatletter %
		\csname colorref@\roman{colorrefCtr}\endcsname%
	\makeatother
}

\newcommand{\sio}{s.\,o.\xspace}
\newcommand{\Sio}{S.\,o.\xspace}
\newcommand{\siu}{s.\,u.\xspace}
\newcommand{\Siu}{S.\,u.\xspace}

\newcommand{\orr}{\vee}
	\newcommand{\bigorr}{\bigvee}

\newcommand{\textmarker}[2]{\sethlcolor{#1}\hl{#2}}

\newcommand{\ua}{u.\,a.\xspace}
\newcommand{\Ua}{U.\,a.\xspace}

\newcommand{\va}{v.\,a.\xspace}
\newcommand{\Va}{V.\,a.\xspace}


\newcommand{\wiki}[1]{\href{#1}{Wikipedia}}

\newcommand{\zb}{z.\,B.\xspace}
\newcommand{\Zb}{Z.\,B.\xspace}

\newcommand{\zt}{z.\,T.\xspace}
\newcommand{\Zt}{Z.\,T.\xspace}


\IfStrEqCase{\dokumentTyp}{
	{Skript}{
		\newcommand{\qed}{\hfill\ensuremath{\square}}
	}
	{Mitschrift}{
		\newcommand{\qed}{\hfill\ensuremath{\square}}
	}
	{Praesentation}{
		\newcommand{\itemz}{\item[\RIGHTarrow]}
	}%
}



% =====================================================
%    Eigene Umgebungen
% =====================================================
\newcommand{\bspT}{}
\newenvironmentx{beispiel}[2][1={}, 2={}]{%
	\renewcommand{\bspT}{#2}
	\paragraph{Beispiel\xspace#1}%
		\begin{quotation}%
			\setlength{\parindent}{0em}\noindent%
			\ifthenelse{\equal{\bspT}{o} \OR \equal{\bspT}{b}}{\vspace*{-2em}}{}%
}{%
	\ifthenelse{\equal{\bspT}{u} \OR \equal{\bspT}{b}}{\vspace*{-1em}}{}%
	\end{quotation}%
}

\newenvironmentx{beispiele}[2][1={}, 2={}]{%
	\renewcommand{\bspT}{#2}
	\paragraph{Beispiele\xspace#1}%
	\begin{quotation}%
		\setlength{\parindent}{0em}\noindent%
		\ifthenelse{\equal{\bspT}{o} \OR \equal{\bspT}{b}}{\vspace*{-2em}}{}%
	}{%
	\ifthenelse{\equal{\bspT}{u} \OR \equal{\bspT}{b}}{\vspace*{-1em}}{}%
	\end{quotation}%
}

\newenvironmentx{beweis}[2][1={}, 2={}]{%
	\renewcommand{\bspT}{#2}
	\paragraph{Beweis\xspace#1}%
	\begin{quotation}%
		\setlength{\parindent}{0em}\noindent%
		\ifthenelse{\equal{\bspT}{o} \OR \equal{\bspT}{b}}{\vspace*{-1em}}{}
	}{%
	\ifthenelse{\equal{\bspT}{u} \OR \equal{\bspT}{b}}{\vspace*{-1.3em}}{}%
	\qed%
	\ifthenelse{\equal{\bspT}{u} \OR \equal{\bspT}{b}}{\vspace*{.5em}}{}%
	\end{quotation}%
}

\newenvironment{enumeratealpha}{\begin{enumerate}[label={\alph*)}]}{\end{enumerate}}

\newenvironment{enumerateAlpha}{\begin{enumerate}[label={\Alph*)}]}{\end{enumerate}}

\newenvironment{enumerateroman}{\begin{enumerate}[label={\roman*)}]}{\end{enumerate}}

\newenvironment{enumerateRoman}{\begin{enumerate}[label={\Roman*)}]}{\end{enumerate}}


\newcommand{\tmpQx}{}
\newcommand{\kHand}[1]{%
	\IfStrEqCase{#1}{%
		{e}{e\hspace*{-.5em}\raisebox{.05em}{\reflectbox{\c{}}}\hspace*{-.2em}}%
		{a}{a\hspace*{-.6em}\raisebox{.05em}{\reflectbox{\c{}}}\hspace*{-.1em}}%
	}[\errmessage{Ogonek nur unter 'a' und 'e'!}\stop]
}
\newcommand{\kFraktur}[1]{%
	\IfStrEqCase{#1}{%
		{e}{e\hspace*{-.1em}\raisebox{.05em}{\reflectbox{\c{}}}\hspace*{-.2em}}%
		{a}{a\hspace*{-.15em}\raisebox{.05em}{\reflectbox{\c{}}}\hspace*{-.15em}}%
	}[\errmessage{Ogonek nur unter 'a' und 'e'!}\stop]
}
\newenvironmentx{quotex}[2][1={},2={}]{%
	\renewcommand{\tmpQx}{#1}
	\begin{quote}%
		\IfStrEqCase{#2}{
			{tt}{ \texttt{#1} }%
			{hand}{ \let\k\kHand \oeschfamily }%
			{fraktur}{  \let\k\kFraktur \frakfamily }%
		}[ \normalfont ]
}{%
		\ifthenelse{\equal{\tmpQx}{}}{}{\par \mbox{} \hfill \normalfont --- \emph{\tmpQx}}%
	\end{quote}%
}
\newenvironmentx{quotationx}[2][1={},2={default}]{%
	\renewcommand{\tmpQx}{#1}
	\begin{quotation}%
		\IfStrEqCase{#2}{
			{tt}{ \texttt{#1} }%
			{hand}{ \oeschfamily }%
			{fraktur}{ \frakfamily }%
			{default}{ \normalfont }%
		}[ \errmessage{Ungueltiger Parameter: \#2 muss 'tt' oder 'hand' oder 'fraktur' sein!}\stop ]
}{%
		\ifthenelse{\equal{\tmpQx}{}}{}{\par \mbox{} \hfill \normalfont --- \emph{\tmpQx}}%
	\end{quotation}%
}


% =====================================================
%    Farbige Boxen (tcolorbox) und Indizes
% =====================================================
\tcbuselibrary{listings,breakable}

\newcommand{\boxnummer}{\thetcbcounter}
\newtcolorbox[auto counter,number format=\arabic,list inside=blueboxes]{blueboxIdx}[2][box\boxnummer]{%
	colframe=blue!75!black,fonttitle=\bfseries,title={#2},phantomlabel={#1},breakable%
}

\newtcolorbox[use counter from=blueboxIdx,number format=\arabic,list inside=yellowboxes]{yellowboxIdx}[2][box\boxnummer]{%
	colframe=yellow!99,fonttitle=\bfseries,title={\textcolor{black}{#2}},phantomlabel={#1},breakable%
}

\newtcolorbox[use counter from=blueboxIdx,number format=\arabic,list inside=greenboxes]{greenboxIdx}[2][box\boxnummer]{%
	colframe=green,fonttitle=\bfseries,title={\textcolor{black}{#2}},phantomlabel={#1},breakable%
}

\newtcolorbox[number format=\arabic]{redbox}[1][]{%
	colframe=red!99,fonttitle=\bfseries,title={\textcolor{black}{Achtung!}},phantomlabel={#1},breakable%
}



% =====================================================
%    Listings einrichten
% =====================================================

\lstset{literate=
	{á}{{\'a}}1 {é}{{\'e}}1 {í}{{\'i}}1 {ó}{{\'o}}1 {ú}{{\'u}}1
	{Á}{{\'A}}1 {É}{{\'E}}1 {Í}{{\'I}}1 {Ó}{{\'O}}1 {Ú}{{\'U}}1
	{à}{{\`a}}1 {è}{{\`e}}1 {ì}{{\`i}}1 {ò}{{\`o}}1 {ù}{{\`u}}1
	{À}{{\`A}}1 {È}{{\'E}}1 {Ì}{{\`I}}1 {Ò}{{\`O}}1 {Ù}{{\`U}}1
	{ä}{{\"a}}1 {ë}{{\"e}}1 {ï}{{\"i}}1 {ö}{{\"o}}1 {ü}{{\"u}}1
	{Ä}{{\"A}}1 {Ë}{{\"E}}1 {Ï}{{\"I}}1 {Ö}{{\"O}}1 {Ü}{{\"U}}1
	{â}{{\^a}}1 {ê}{{\^e}}1 {î}{{\^i}}1 {ô}{{\^o}}1 {û}{{\^u}}1
	{Â}{{\^A}}1 {Ê}{{\^E}}1 {Î}{{\^I}}1 {Ô}{{\^O}}1 {Û}{{\^U}}1
	{œ}{{\oe}}1 {Œ}{{\OE}}1 {æ}{{\ae}}1 {Æ}{{\AE}}1 {ß}{{\ss}}1
	{ű}{{\H{u}}}1 {Ű}{{\H{U}}}1 {ő}{{\H{o}}}1 {Ő}{{\H{O}}}1
	{ç}{{\c c}}1 {Ç}{{\c C}}1 {ø}{{\o}}1 {å}{{\r a}}1 {Å}{{\r A}}1
	{€}{{\EUR}}1 {£}{{\pounds}}1 {~}{{\textasciitilde}}1
}

\definecolor{myGray}{RGB}{220,225,225}

\lstset{
	backgroundcolor=\color{myGray},
	numbers=left,
	breakautoindent=true,
	tabsize=4,
	frame=single,
	breaklines=true,
	postbreak=\raisebox{0ex}[0ex][0ex]{\ensuremath{\color{red}\hookrightarrow\space}},
	caption=\texttt\lstname
}

\definecolor{javared}{rgb}{0.6,0,0} % for strings
\definecolor{javagreen}{rgb}{0.25,0.5,0.35} % comments
\definecolor{javapurple}{rgb}{0.5,0,0.35} % keywords
\definecolor{javadocblue}{rgb}{0.25,0.35,0.75} % javadoc
\definecolor{javablue}{RGB}{0,37,189} % javadoc

\lstset{language=Java,
	basicstyle=\ttfamily,
	keywordstyle=\color{javapurple}\bfseries,
	stringstyle=\color{javablue},
	commentstyle=\color{javagreen},
	morecomment=[s][\color{javadocblue}]{/**}{*/},
	numbers=left,
	numberstyle=\tiny\color{black},
	numbersep=10pt,
	tabsize=4,
	showspaces=false,
	showstringspaces=false
}

% =====================================================
%    Programmierdokumentation
% =====================================================
\definecolor{eclipsemodifier}{RGB}{143,0,83}
\definecolor{eclipsevar}{RGB}{0,37,189}
\newcommand{\documentationxMargin}{15pt}
\newenvironment{documentationx}{
	\fboxsep1mm
	\newcommand{\Parameter}[2][]{%
		\ifthenelse%
			{\equal{##1}{}}%
			{}%
			{%
				\IfStrEqCase{##1}{%
					{boolean}{\textcolor{eclipsemodifier}{##1}}%
					{bool}{\textcolor{eclipsemodifier}{##1}}%
					{byte}{\textcolor{eclipsemodifier}{##1}}%
					{char}{\textcolor{eclipsemodifier}{##1}}%
					{double}{\textcolor{eclipsemodifier}{##1}}%
					{short}{\textcolor{eclipsemodifier}{##1}}%
					{int}{\textcolor{eclipsemodifier}{##1}}%
					{float}{\textcolor{eclipsemodifier}{##1}}%
				}[##1]%
				\ %
			}%
		\textcolor{eclipsevar}{##2}%
	}

	\newcommand{\Pkgdef}[2]{%
		\pgfqkeys{/documentationx/pkg/##1}{
			name/.initial = {},
			description/.initial = {},
		}%
		\pgfqkeys{/documentationx/pkg/##1}{##2}%
		\noindent%
		\fbox{\textcolor{eclipsemodifier}{\texttt{package}} \texttt{\pgfkeysvalueof{/documentationx/pkg/##1/name}}}:
		\begin{addmargin}[\documentationxMargin]{0pt}
			\pgfkeysvalueof{/documentationx/pkg/##1/description}
		\end{addmargin}
	}
	\newcommand{\Classdef}[2]{%
		\pgfqkeys{/documentationx/class/##1}{
			name/.initial = {},
			description/.initial = {},
			modifier/.initial = {},
			type/.initial = {}
		}%
		\pgfqkeys{/documentationx/class/##1}{##2}%
		\noindent%
		\begingroup
			\texttt{
				\ifthenelse%
					{\equal{}{\pgfkeysvalueof{/documentationx/class/##1/modifier}}{}}%
					{}%
					{\textcolor{eclipsemodifier}{\pgfkeysvalueof{/documentationx/class/##1/modifier}}\ }%
				\ifthenelse%
					{\equal{}{\pgfkeysvalueof{/documentationx/class/##1/type}}{}}%
					{}%
					{\textcolor{eclipsemodifier}{\pgfkeysvalueof{/documentationx/class/##1/type}}\ }%
				\pgfkeysvalueof{/documentationx/class/##1/name}%
			}%
	%	\ifthenelse%
	%		{\equal{\pgfkeysvalueof{/documentationx/class/##1/description}}{}}%
	%		{}%
	%		{%
				\normalfont :
				\begin{addmargin}[\documentationxMargin]{0pt}
					\pgfkeysvalueof{/documentationx/class/##1/description}
				\end{addmargin}
	%		}%
		\endgroup%
	}
	\newcommand{\Vardef}[2]{%
		\pgfqkeys{/documentationx/var/##1}{
			name/.initial = {},
			description/.initial = {},
			modifier/.initial = {},
			type/.initial = {}
		}%
		\pgfqkeys{/documentationx/var/##1}{##2}%
		\noindent%
		\begingroup
			\texttt{%
				\ifthenelse%
					{\equal{}{\pgfkeysvalueof{/documentationx/var/##1/modifier}}{}}%
					{}%
					{\textcolor{eclipsemodifier}{\pgfkeysvalueof{/documentationx/var/##1/modifier}}\ }%
				\ifthenelse%
					{\equal{}{\pgfkeysvalueof{/documentationx/var/##1/type}}{}}%
					{}%
					{\textcolor{eclipsemodifier}{\pgfkeysvalueof{/documentationx/var/##1/type}}\ }%
				\textcolor{eclipsevar}{\pgfkeysvalueof{/documentationx/var/##1/name}}%
				\ifthenelse%
					{\equal{\pgfkeysvalueof{/documentationx/var/##1/description}}{}}%
					{}%
					{%
						\normalfont :
						\begin{addmargin}[\documentationxMargin]{0pt}
							\pgfkeysvalueof{/documentationx/var/##1/description}
						\end{addmargin}
					}%
			}%
		\endgroup%
	}
	\newcommand{\Methoddef}[2]{%
		\pgfqkeys{/documentationx/method/##1}{
			name/.initial = {},
			description/.initial = {},
			modifier/.initial = {},
			type/.initial = {} % return type
		}
		\pgfqkeys{/documentationx/method/##1}{##2}%
		\noindent%
		\begingroup
			\texttt{
				\ifthenelse%
					{\equal{}{\pgfkeysvalueof{/documentationx/method/##1/modifier}}{}}%
					{}%
					{\textcolor{eclipsemodifier}{\pgfkeysvalueof{/documentationx/method/##1/modifier}}\ }%
				\ifthenelse%
					{\equal{}{\pgfkeysvalueof{/documentationx/method/##1/type}}{}}%
					{}%
					{\textcolor{eclipsemodifier}{\pgfkeysvalueof{/documentationx/method/##1/type}}\ }%
				\pgfkeysvalueof{/documentationx/method/##1/name}%
				\ifthenelse%
					{\equal{\pgfkeysvalueof{/documentationx/method/##1/description}}{}}%
					{}%
					{%
						\normalfont :
						\begin{addmargin}[\documentationxMargin]{0pt}
							\pgfkeysvalueof{/documentationx/method/##1/description}
						\end{addmargin}
					}%
			}%
		\endgroup%
	}
}

% =====================================================
%    Initiale Konfiguration
% =====================================================

\clubpenalty = 10000

\widowpenalty = 10000

\displaywidowpenalty = 10000

\makeatletter
	\@removefromreset{footnote}{chapter}
\makeatother

% Paragraphenüberschriften immer mit einem Punkt beenden. Kann durch die Stern-Variante "\paragraph*{title}" unterdrückt werden.
\renewcommand\paragraphmark[1]{.}

\let\epsilonOld\epsilon
\let\epsilon\varepsilon

\let\phiOld\phi
\let\phi\varphi


% =====================================================
%    Deklaration der Variablen
% =====================================================

\pgfqkeys{/VorlageVersion5}{
	all/Autoren/.initial					= {},
	all/Titel/.initial	 					= {},
	all/Untertitel/.initial 				= {},
	all/VersionPraefix/.initial 			= {},
	all/Version/.initial 					= {true},
	all/Icon/Breite/.initial				= {},
	all/Icon/URL/.initial					= {},
	all/Index/Boxen/blau/titel/.initial 	= {Liste der Sätze und Definitionen},
	all/Index/Boxen/blau/zeigen/.initial 	= {false},
	all/Index/Boxen/gelb/titel/.initial 	= {Liste der Beispiele},
	all/Index/Boxen/gelb/zeigen/.initial 	= {false},
	all/Index/Boxen/gruen/titel/.initial 	= {Liste der Fragen},
	all/Index/Boxen/gruen/zeigen/.initial 	= {false},
	all/Index/Begriffe/titel/.initial 		= {Index},
	all/Index/Begriffe/zeigen/.initial 		= {false},
	all/Index/Literatur/titel/.initial 		= {Literaturverzeichnis},
	all/Index/Literatur/zeigen/.initial 	= {false},
	Skript/AnmerkungenTitelseite/.initial 	= {},
	Mitschriften/Vorlesungsname/.initial 	= {},
	Mitschriften/Typ/.initial 				= {},
	Mitschriften/LfdNr/.initial 			= {},
	Mitschriften/Gruppe/.initial 			= {},
	Mitschriften/Headerhoehe/.initial		= {42pt},
	Buecher/Widmung/.initial				= {},
	Buecher/Modus/.initial					= {rl},
	Praesentationen/TitelKurz/.initial		= {},
	Praesentationen/Institut/lang/.initial	= {},
	Praesentationen/Institut/kurz/.initial	= {},
	Thesis/AkademischerGrad/kurz/.initial	= {},
	Thesis/AkademischerGrad/lang/.initial	= {},
	Thesis/Fach/Nominativ/.initial			= {},
	Thesis/Fach/Genitiv/.initial			= {},
	Thesis/UniName/lang/.initial			= {},
	Thesis/UniName/kurz/.initial			= {},
	Thesis/EingereichtBei/.initial			= {},
	Thesis/BetreuungDurch/.initial			= {},
	Thesis/Abgabedatum/.initial				= {},
	Thesis/Versicherung/zeigen/.initial		= {true},
	Thesis/Versicherung/text/.initial		= {},
	Thesis/Versicherung/ort/.initial		= {}
}


% =====================================================
%    Generiere passende Titelseiten, Kopfzeilen etc.
% =====================================================
\AtBeginDocument{
	% Zulässige Werte testen
	\begingroup
		\newcommand{\checkbool}[1]{\ifthenelse{\equal{\get{#1}}{true} \OR \equal{\get{#1}}{false}}{}{\errmessage{Boolean '#1' muss 'true' oder 'false' sein!}\stop}}
		\checkbool{all/Version}
		\checkbool{all/Index/Boxen/blau/zeigen}
		\checkbool{all/Index/Boxen/gelb/zeigen}
		\checkbool{all/Index/Boxen/gruen/zeigen}
		\checkbool{all/Index/Begriffe/zeigen}
		\checkbool{all/Index/Literatur/zeigen}
		\checkbool{Thesis/Versicherung/zeigen}
		% ------------------------------------------------------------------------
	\endgroup

	\setVersion{\get{all/VersionPraefix}}
	\IfStrEqCase{\dokumentTyp}{
		{Skript}{%
			\thispagestyle{empty}
			\vspace*{4em}
			\begin{center}
				\ifthenelse{\equal{\get{all/Titel}}{}}{}{{\Huge \textbf{\get{all/Titel}}\par}}
				\ifthenelse{\equal{\get{all/Untertitel}}{}}{}{\vspace*{.5em}{\large\textbf{\get{all/Untertitel}}\par}}
			\end{center}
			\begin{flushright}
				\ifthenelse{\equal{\get{Skript/AnmerkungenTitelseite}}{}}{}{{\em \vspace*{1em}\get{Skript/AnmerkungenTitelseite}\par}}
				\ifthenelse{\equal{\get{all/Version}}{true}}{Version: \version, }{}\today\par\vspace*{1em}
				\StrSubstitute{\get{all/Autoren}}{&}{\par}
			\end{flushright}
			\begin{center}
				\IfFileExists{titel.png}{%
					\includegraphics[width=\get{all/Icon/Breite}\textwidth]{titel.png}\par%
					\ifthenelse{\equal{\get{all/Icon/URL}}{}}{}{{\tiny \href{\get{all/Icon/URL}}{Quelle}}}
				}{}
			\end{center}
			\newpage

			\pagenumbering{Roman}
			\newpage
			\tableofcontents

			\newpage

			\newpage
			\pagenumbering{arabic}
		}
		{Thesis}{%
			\ifthenelse{\equal{\get{all/Autoren}}{}}{\errmessage{Kein Autor bei der Thesis angegeben!}\stop}{}
			\ifthenelse{\equal{\get{all/Titel}}{}}{\errmessage{Kein Titel bei der Thesis angegeben!}\stop}{}

			% an anderer Stelle führt dies zu einem Stacküberlauf ("Tex capacity exceeded, sorry...")!
			% \newcommand\eingereichtBei{}
			% \StrSubstitute{\get{Thesis/EingereichtBei}}{&}{,\par}[\eingereichtBei]
			% \newcommand\betreuungDurch{}
			% \StrSubstitute{\get{Thesis/BetreuungDurch}}{&}{,\par}[\betreuungDurch]

			\thispagestyle{empty}
			\vspace*{4em}
			\begin{center}
				%\ifthenelse{\equal{\get{Thesis/AkademischerGrad/kurz}}{}}{}{%
					%\get{Thesis/AkademischerGrad/kurz}arbeit zur Erlangung des akademischen Grades\par	eines \get{Thesis/AkademischerGrad/lang} \get{Thesis/Fach/Genitiv}%
					%Abschlussarbeit im Fach \get{Thesis/Fach/Nominativ}%
				%}
				%\\
				\ifthenelse{\equal{\get{Thesis/UniName/lang}}{}}{}{\textbf{\get{Thesis/UniName/lang}}\par}
				\IfFileExists{titel.jpg}{%
					\vspace*{1em}%
					\includegraphics[width=\get{all/Icon/Breite}\textwidth]{titel.jpg}\par%
					\ifthenelse{\equal{\get{all/Icon/URL}}{}}{}{{\tiny \href{\get{all/Icon/URL}}{Quelle}}}
				}{}
				\ifthenelse{\equal{\get{all/Titel}}{}}{}{{\Huge \textbf{\get{all/Titel}}\par}}
				\ifthenelse{\equal{\get{all/Untertitel}}{}}{}{\vspace*{.5em}{\large\textbf{\get{all/Untertitel}}\par}}
				\vspace*{.5em}
				{\textbf{\Large \get{all/Autoren}}}
			\end{center}

			\begin{tabular}{ll}%
				\ifthenelse{\equal{\eingereichtBei}{}}{}{\noindent Eingereicht bei: & \multicolumn{1}{p{11cm}}{\eingereichtBei} \\}%
				\ifthenelse{\equal{\betreuungDurch}{}}{}{\noindent Betreuung durch: & \multicolumn{1}{p{11cm}}{\betreuungDurch} \\}%
				\noindent Abgabedatum:	 & \get{Thesis/Abgabedatum}%
			\end{tabular}
			\begin{flushright}
				\ifthenelse{\equal{\get{Skript/AnmerkungenTitelseite}}{}}{}{{\em \vspace*{1em}\get{Skript/AnmerkungenTitelseite}\par}}
				\ifthenelse{\equal{\get{all/Version}}{true}}{Version: \version}{}
			\end{flushright}

			\newpage

			\pagenumbering{Roman}
			\newpage
			\tableofcontents

			\newpage

			\newpage
			\pagenumbering{arabic}
		}
		{Mitschrift}{
			\setlength{\parindent}{0cm}
			\setlength{\headheight}{\get{Mitschriften/Headerhoehe}}
			\pagestyle{fancy}
			\newcommand{\autorTemp}{\StrSubstitute{\get{all/Autoren}}{&}{\par}}
			\expandafter\lhead{\autorTemp}
			\chead{%
				\IfFileExists{titel.jpg}{\includegraphics[width=\get{all/Icon/Breite}\textwidth]{titel.jpg}\\}{}
				\ifthenelse{\equal{\get{Mitschriften/Gruppe}}{}}{}{Gruppe \get{Mitschriften/Gruppe}}
			}
			\rhead{\textbf{\emph{\get{Mitschriften/Vorlesungsname}}}\\\textbf{\get{Mitschriften/Typ}\ \get{Mitschriften/LfdNr}}\ifthenelse{\equal{\get{all/Version}}{true}}{\\Stand: \today}{}}
			\lfoot{}
			\cfoot{-- \thepage\ --}
			\rfoot{}

			\global\long\def\headrulewidth{0.4pt}
			\global\long\def\footrulewidth{0.4pt}
			\ifthenelse{\equal{\get{all/Titel}}{}}%
				{\section*{\get{Mitschriften/Typ}\ \get{Mitschriften/LfdNr}}}%
				{\section*{\get{all/Titel}}}

			\ifthenelse{\equal{\get{all/Untertitel}}{}}{}{\subsubsection*{\get{all/Untertitel}}}
		}
		{Praesentation}{
			\title{\get{all/Titel}}
			\subtitle{\get{all/Untertitel}}
			\StrSubstitute{\get{all/Autoren}}{&}{\par}[\autorTemp]
			\author{\autorTemp}
			\IfFileExists{titel.jpg}{\logo{titel.jpg}}{}
			\ifthenelse{\equal{\get{Praesentationen/Institut}}{}}{}{\institute{\get{Praesentationen/Institut}}}

			\frame{\titlepage}

			\frame{
				\finp
				\tableofcontents
			}
		}
	}
}

\AtEndDocument{
	\appendix


		\renewcommand{\bibname}{\get{all/Index/Literatur/titel}}
		\bibliographystyle{alphadin}
		\nocite{*}
		\bibliography{quellen}


	\IfStrEqCase{\dokumentTyp}{
		{Skript}{%
			\ifthenelse{\equal{\get{all/Index/Boxen/blau/zeigen}}{true}}{\tcblistof[\chapter]{blueboxes}{\get{all/Index/Boxen/blau/titel}}}{}

			\ifthenelse{\equal{\get{all/Index/Boxen/gelb/zeigen}}{true}}{\tcblistof[\chapter]{yellowboxes}{\get{all/Index/Boxen/gelb/titel}}}{}

			\ifthenelse{\equal{\get{all/Index/Boxen/gruen/zeigen}}{true}}{\tcblistof[\chapter]{greenboxes}{\get{all/Index/Boxen/gruen/titel}}}{}

			\newpage
			\ifthenelse{\equal{\get{all/Index/Begriffe/zeigen}}{true}}{
				\renewcommand{\indexname}{\get{all/Index/Begriffe/titel}}
				\phantomsection
				\addcontentsline{toc}{chapter}{\indexname}
				\printindex
			}{}
		}
		{Thesis}{%
			\ifthenelse{\equal{\get{Thesis/Versicherung/zeigen}}{true}}{%
				\chapter{Eigenständigkeitserklärung}%
				\ifthenelse{\equal{\get{Thesis/Versicherung/text}}{}}%
					{Ich erkläre hiermit, dass ich die vorliegende Arbeit selbständig verfasst, keine anderen als die angegebenen Quellen und Hilfsmittel benutzt sowie die diesen Quellen und Hilfsmitteln wörtlich oder sinngemäß entnommenen Ausführungen als solche kenntlich gemacht habe. \par
					\noindent Die Arbeit habe ich bisher oder gleichzeitig keiner anderen Prüfungsbehörde vorgelegt.
					\\ \\
				}{\get{Thesis/Versicherung/text}}%
				\noindent \get{Thesis/Versicherung/ort}, \get{Thesis/Abgabedatum}
				\\ \\ \\

				\noindent ......................................................

				\noindent \StrSubstitute{\get{all/Autoren}}{&}{\par}

				\newpage%
			}{}

			\ifthenelse{\equal{\get{all/Index/Boxen/blau/zeigen}}{true}}{\tcblistof[\chapter]{blueboxes}{\get{all/Index/Boxen/blau/titel}}}{}

			\ifthenelse{\equal{\get{all/Index/Boxen/gelb/zeigen}}{true}}{\tcblistof[\chapter]{yellowboxes}{\get{all/Index/Boxen/gelb/titel}}}{}

			\ifthenelse{\equal{\get{all/Index/Boxen/gruen/zeigen}}{true}}{\tcblistof[\chapter]{greenboxes}{\get{all/Index/Boxen/gruen/titel}}}{}

			\ifthenelse{\equal{\get{all/Index/Begriffe/zeigen}}{true}}{
				\renewcommand{\indexname}{\get{all/Index/Begriffe/titel}}
				\addcontentsline{toc}{chapter}{\indexname}
				\printindex
			}{}
		}
		{Mitschrift}{%
			\ifthenelse{\equal{\get{all/Index/Boxen/blau/zeigen}}{true}}{\tcblistof[\chapter]{blueboxes}{\get{all/Index/Boxen/blau/titel}}}{}

			\ifthenelse{\equal{\get{all/Index/Boxen/gelb/zeigen}}{true}}{\tcblistof[\chapter]{yellowboxes}{\get{all/Index/Boxen/gelb/titel}}}{}

			\ifthenelse{\equal{\get{all/Index/Boxen/gruen/zeigen}}{true}}{\tcblistof[\chapter]{greenboxes}{\get{all/Index/Boxen/gruen/titel}}}{}

			\ifthenelse{\equal{\get{all/Index/Begriffe/zeigen}}{true}}{
				\renewcommand{\indexname}{\get{all/Index/Begriffe/titel}}
				\addcontentsline{toc}{chapter}{\indexname}
				\printindex
			}{}
		}
		{Praesentation}{%
			\ifthenelse{\equal{\get{all/Index/Begriffe/zeigen}}{true}}{
				\newenvironment{theindex}{\let\item\par}{}

				\newcommand\indexspace{}

				\begin{frame}
					\frametitle{\get{all/Index/Begriffe/titel}}
					\printindex
				\end{frame}
			}{}
		}
	}
}

