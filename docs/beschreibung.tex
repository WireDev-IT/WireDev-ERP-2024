Ziel ist es, im Rahmen der Klasse 13, ein Softwareprodukt zu entwickeln, welches Unternehmen im Einkauf und Verkauf von Produkte nützlich ist. Die Anwendung soll in 90 Tagen einsatzbereit sein.

\section{C\#}
C\# (gesprochen „C-Sharp“) ist eine objektorientierte Programmiersprache, die im Auftrag von Microsoft entwickelt und 2001 veröffentlicht wurde. Die Sprache ist grundsätzlich plattformunabhängig und wurde für die Softwareplattform .NET entwickelt. C\# ist auch die Sprache, die in diesem Projekt ihren Einsatz findet.

\section{HTTP}
HTTP steht für „Hypertext Transfer Protocol“ und ist das Kommunikationsprotokoll im World Wide Web (WWW). Es ist für das Abrufen von statischen Inhalten vorgesehen. Dabei lädt der Client Dateien von dem Server herunter. Diese werden aber nicht dauerhaft, sondern nur im Cache (Pufferspeicher) gespeichert und später wieder verworfen. HTTPS (HTTP Secure) bietet eine zusätzliche Verschlüsselung. Da es vorgesehen ist, sensible Daten zu übertragen, wird es hier auch verwendet werden.

\section{IDE}
Für eine effiziente Arbeitsweise benötige ich Programme, welche mir beim Schreiben des Quelltextes helfen und diesen für mich kompilieren, damit am Ende eine ausführbare Anwendung bereitsteht. Auch hier greife ich wieder auf eine mir bereits bekannte Lösung von Microsoft zurück: Microsoft Visual Studio 2022. Das Programm bietet hilfreiche Funktionen wie das automatische Installieren von Paketen, das farbige Markieren von Quelltexten für eine bessere Lesbarkeit und einen Kompilierer, der C-Sharp versteht. Die IDE gibt mir außerdem Möglichkeiten in meine Anwendung während der Laufzeit hereinzuschauen, um Fehler schneller finden zu können sowie die Korrektheit der Vorgänge zu überprüfen.

\section{Git}
Zur Sicherung verschiedener Versionen meiner geplanten App, das heißt zur Versionsverwaltung, werde ich GitHub, einen Cloud-basierten Git Repository Hosting Service Anbieter, nutzen. Über Git lassen sich Versionsverläufe erstellen. GitHub erweitert die Funktionalität mit einem sicheren Speicherort, Werkzeugen zur Planung des Projekts und kollaborativen Möglichkeiten für Entwicklerteams. Mit dieser Plattform werde ich meine Aufgaben und Ideen gliedern und umsetzen.

\section{SQLite}
Das relationale Datenbank-Management-System (DBMS) ist eine leichte, Serverlose Lösung für eine schnelle Datenbankanbindung. Die Open-Source-Software kann in Anwendungen integriert werden, um Speicherung von Daten ohne separaten Datenbankserver zu ermöglichen. SQLite wurde in C geschrieben und ist mit vielen Betriebssystem wie Windows, Linux, MacOS, Android und iOS. Unterstützt werden grundlegende SQL-Operationen wie INSERT, UPDATE; SELECT und DELETE. Abfragen mit Aggregatfunktionen, Unterabfragen und Joins sind ebenso möglich. Durch die ACID-Konformität ist Zuverlässigkeit und Robustheit gewährleistet.