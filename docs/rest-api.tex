Der Architekturstil des \emph{Representational State Transfer Application Programming Interface} (REST-API) wurde für Webanwendungen entwickelt, welche eine Schnittstelle zwischen verschiedenen Endpunkten implementieren. 
Diese dienen dem Austausch von Daten und Ressourcen. 
Um Zugriff auf diese zu erhalten, stellt das HTTP-Protokoll verschiedene Methoden (\emph{HTTP-Verben}) zuer Verfügung. Die wichtigsten sind \texttt{GET}, \texttt{POST}, \texttt{PUT} und \texttt{DELETE}. 
Zur Adressierung werden URLs (Uniform  Resource Locator), eine spezielle Form von URIs (Uniform Resource Identifier) verwendet.

Anfragen an die API erfolgen üblicherweise im JSON- oder XML-Format, um die angeforderten oder übertragenen Daten zu gliedern. 
Die oben genannten Verben haben folgende Bedeutungen:
\begin{itemize}
	\item \texttt{GET}: Abrufen von Daten aus einer Ressource
	\item \texttt{POST}: Erstellen von Daten auf der Ressource
	\item \texttt{PUT}: Aktualisieren von Daten auf der Ressource
	\item \texttt{DELETE}: Löschen von Daten von der Ressource
\end{itemize}

In diesem Projekt werden API-Endpunkte (gekapselt in sog. \emph{Controllern}) und andere Funktionen wie z.\,B. Datenbankoperationen voneinander getrennt.
Das verwendete HTTP-Framework bindet den HTTP-Request an eine Funktion, wodurch eine einfache und saubere Implementierung ermöglicht wird.
Somit können Endpunkte einfach verschoben und Integration Tests einfacher durchgeführt werden.
Annotationen mit den HTTP-Verben definieren dabei die Art der Anfrage.
Das ermöglicht, die Zugehörigkeit direkt im Quelltext abzulesen.

Die in der URL übertragenen Parameter werden direkt die der Funktion zugeordnet.
Dadurch wird sichergestellt, das die Daten aus dem Body der Anfrage dem Datentyp entsprechen, den die Funktion verarbeiten kann.
Sollte dies nicht der Fall sein, beantwortet das Framework, welches in Abschnitt \ref{backend} genauer erklärt wird, die Anfrage mit einem HTTP-Fehlercode.

Funktion geben immer ein \texttt{IActionResult} zurück, welches die Klasse \texttt{ObjectResult} enthält.
Die hier am Wichtigsten enthaltenen Eigenschaften sind der Statuscode und der Body.
Das Framework kann somit eine HTTP-Antwort zurückgeben und gleichzeitig können Integration Tests intern, d.\,h. ohne HTTP, durchgeführt werden.