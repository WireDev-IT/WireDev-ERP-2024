Der Architekturstil von Representational State Transfer Application Programming Interface (REST-API) wurde für Webanwendungen entwickelt, welche eine Schnittstelle zwischen verschiedenen Endpunkten implementieren.
Diese dienen dem Austausch von Daten und Ressourcen.
Um Zugriff auf diese zu erhalten, wird das Prinzip der Nutzung von HTTP verwendet: GET, POST, PUT, DELETE.
Zur Adressierung werden URI (Uniform Resource Identifier) und URL (Uniform Resource Locator) verwendet.

Anfragen an die API erfolgen gängigerweise im JSON- oder XML-Format, um die angeforderten oder übertragenen Daten zu gliedern.
Zur Manipulation werden 4 Anfragetypen verwendet:

\begin{itemize}
	\item GET: Abrufen von Daten aus einer Ressource
	\item POST: Erstellen von Daten auf der Ressource
	\item PUT: Aktualisieren von Daten auf der Ressource
	\item DELETE: Löschen von Daten von der Ressource
